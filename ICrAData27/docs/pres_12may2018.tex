%%%
%%% Presentation for May 12, 2018
%%%

% http://texblog.org/2008/01/21/create-your-slides-presentations-with-latex/
% http://tex.stackexchange.com/questions/24035/typeset-mathematical-symbols-also-in-sans-serif-font
% http://tex.stackexchange.com/questions/83545/int-with-limits-or-without
% http://yihui.name/en/2010/08/numbered-figure-table-captions-in-beamer/

% detokenize - http://tex.stackexchange.com/questions/48632/underscores-in-words-text

%%% Трябва ни текст на кирилица за UTF8
%%% Така TeXLive използва шрифтове cm-super, които изглеждат по-добре


\documentclass{beamer}

\usepackage{cmap}
\usepackage[T1, T2A]{fontenc}
\usepackage[utf8]{inputenc}
\usepackage[bulgarian, english]{babel}

\usepackage{amsmath}
\usepackage{amssymb}
\usepackage{amsthm}
\usepackage{mathrsfs}
\usepackage{graphicx}


\usetheme{Madrid}
\beamertemplatenavigationsymbolsempty

\setbeamertemplate{caption}[numbered]
% \setbeamertemplate{bibliography item}{\insertbiblabel}


% \newcommand{\inclgraph}{\includegraphics[scale=0.9]}
% \newcommand{\inclgraphtwo}{\includegraphics[scale=0.85]}
% \newcommand{\inclgraphthree}{\includegraphics[scale=0.35]}

% \newcommand{\inclgraph}{\detokenize}
% \newcommand{\inclgraphtwo}{\detokenize}
% \newcommand{\inclgraphthree}{\detokenize}

\definecolor{red}{HTML}{FF0000}


\begin{document}


\title[ICrAData -- Software for ICrA]{ICrAData -- Software for InterCriteria Analysis}
\author[N. Ikonomov]{Nikolay Ikonomov}
\institute[]{Institute of Mathematics and Informatics\\
Bulgarian Academy of Sciences}
\date{May 12, 2018}

\begin{frame}
\maketitle
\end{frame}


\begin{frame}
InterCriteria Analysis \cite{amv2014} is based on Index Matrices \cite{atanassov2014}
and Intuitionistic Fuzzy Sets \cite{atanassov2012}.

\bigskip
Let an index matrix be given, where $O_n$ are the objects, and $C_n$ are the criteria by which the objects are evaluated:
\begin{center}
\begin{tabular}{c|cccc}
& $O_1$ & $O_2$ & \ldots & $O_n$ \\
\hline $C_1$ & $C_1(O_1)$ & $C_1(O_2)$ & \ldots & $C_1(O_n)$ \\
$C_2$ & $C_2(O_1)$ & $C_2(O_2)$ & \ldots & $C_2(O_n)$ \\
\vdots & \vdots & \vdots & $\ddots$ & \vdots \\
$C_m$ & $C_m(O_1)$ & $C_m(O_2)$ & \ldots & $C_m(O_n)$
\end{tabular}
\end{center}

The criteria matrix, created from the index matrix, is:
\begin{center}
\begin{tabular}{c|cccc}
& & & & \\
\hline $C_1$ & {\scriptsize $C_1(O_1)-C_1(O_2)$} & {\scriptsize $C_1(O_1)-C_1(O_3)$} \ldots &
	{\scriptsize $C_1(O_1)-C_1(O_n)$} & {\scriptsize $C_1(O_2)-C_1(O_3)$}\ldots \\
$C_2$ & {\scriptsize $C_2(O_1)-C_2(O_2)$} & {\scriptsize $C_2(O_1)-C_2(O_3)$} \ldots &
	{\scriptsize $C_2(O_1)-C_2(O_n)$} & {\scriptsize $C_2(O_2)-C_2(O_3)$}\ldots \\
\vdots & \vdots & \vdots & $\ddots$ & \vdots \\
$C_n$ & {\scriptsize $C_n(O_1)-C_n(O_2)$} & {\scriptsize $C_n(O_1)-C_n(O_3)$} \ldots &
	{\scriptsize $C_n(O_1)-C_n(O_n)$} & {\scriptsize $C_n(O_2)-C_n(O_3)$}\ldots
\end{tabular}
\end{center}
\end{frame}


\begin{frame}
Let's demonstrate the algorithm with specific values:
\begin{center}
\begin{tabular}{c|ccccc}
& $O_1$ & $O_2$ & $O_3$ & $O_4$ & $O_5$ \\
\hline $C_1$ & 6 & 5 & 3 & 7 & 6 \\
$C_2$ & 7 & 7 & 8 & 1 & 3 \\
$C_3$ & 4 & 3 & 5 & 9 & 1 \\
$C_4$ & 4 & 5 & 6 & 7 & 8 \\
\end{tabular}
\end{center}

The criteria matrix is:
\begin{center}
\begin{tabular}{c|cccccccccc}
& {\tiny (1-2)} & {\tiny (1-3)} & {\tiny (1-4)} & {\tiny (1-5)} &
{\tiny (2-3)} & {\tiny (2-4)} & {\tiny (2-5)} & {\tiny (3-4)} & {\tiny (3-5)} & {\tiny (4-5)} \\
\hline $C_1$ & 1 & 3 & -1 & 0 & 2 & -2 & -1 & -4 & -3 & 1 \\
$C_2$ & 0 & -1 & 6 & 4 & -1 & 6 & 4 & 7 & 5 & -2 \\
$C_3$ & 1 & -1 & -5 & 3 & -2 & -6 & 2 & -4 & 4 & 8 \\
$C_4$ & -1 & -2 & -3 & -4 & -1 & -2 & -3 & -1 & -2 & -1
\end{tabular}
\end{center}
\end{frame}


\begin{frame}
We create a new matrix, which takes only the sign from each value of the criteria matrix:
\begin{center}
\begin{tabular}{c|cccccccccc}
& & & & & & & & & & \\
\hline $S_1$ & 1 & 1 & -1 & 0 & 1 & -1 & -1 & -1 & -1 & 1 \\
$S_2$ & 0 & -1 & 1 & 1 & -1 & 1 & 1 & 1 & 1 & -1 \\
$S_3$ & 1 & -1 & -1 & 1 & -1 & -1 & 1 & -1 & 1 & 1 \\
$S_4$ & -1 & -1 & -1 & -1 & -1 & -1 & -1 & -1 & -1 & -1
\end{tabular}
\end{center}

The final matrix (which is the result) is obtained by comparing each row with all rows of the criteria matrix:
\begin{center}
\begin{tabular}{c|cccc}
& $C_1$ & $C_2$ & $C_3$ & $C_4$ \\
\hline $C_1$ & $S_1\#S_1$ & $S_1\#S_2$ & $S_1\#S_3$ & $S_1\#S_4$ \\
$C_2$ & - & $S_2\#S_2$ & $S_2\#S_3$ & $S_2\#S_4$ \\
$C_3$ & - & - & $S_3\#S_3$ & $S_3\#S_4$ \\
$C_4$ & - & - & - & $S_4\#S_4$
\end{tabular}
\end{center}
\end{frame}


\begin{frame}
Method \textbf{$\mu$-biased}.

\bigskip
We use these comparisons for matrix $\mu$: $0=0$, $1=1$, $-1=-1$.\\
Also these comparisons for matrix $\nu$: $-1\neq1$, $1\neq-1$.

\begin{center}
\begin{tabular}{c|cccccccccc}
& & & & & & & & & & \\
\hline $S_1$ & 1 & 1 & -1 & 0 & 1 & -1 & -1 & -1 & -1 & 1 \\
$S_2$ & 0 & -1 & 1 & 1 & -1 & 1 & 1 & 1 & 1 & -1 \\
$S_3$ & 1 & -1 & -1 & 1 & -1 & -1 & 1 & -1 & 1 & 1 \\
$S_4$ & -1 & -1 & -1 & -1 & -1 & -1 & -1 & -1 & -1 & -1
\end{tabular}
\end{center}

\begin{center}
\begin{minipage}[b]{0.4\linewidth}
\begin{tabular}{c|cccc}
$\mu$ & $C_1$ & $C_2$ & $C_3$ & $C_4$ \\
\hline $C_1$ & 1 & 0 & 0.5 & 0.5 \\
$C_2$ & - & 1 & 0.5 & 0.3 \\
$C_3$ & - & - & 1 & 0.5 \\
$C_4$ & - & - & - & 1
\end{tabular}
\end{minipage}
\begin{minipage}[b]{0.4\linewidth}
\begin{tabular}{c|cccc}
$\nu$ & $C_1$ & $C_2$ & $C_3$ & $C_4$ \\
\hline $C_1$ & 0 & 0.8 & 0.4 & 0.4 \\
$C_2$ & - & 0 & 0.4 & 0.6 \\
$C_3$ & - & - & 0 & 0.5 \\
$C_4$ & - & - & - & 0
\end{tabular}
\end{minipage}
\end{center}

We count the equal elements (for matrix $\mu$) between two rows, and then we divide by the number of columns.
Non-equal elements for matrix $\nu$.
\end{frame}


\begin{frame}
Method \textbf{Unbiased}.

\bigskip
Comparisons for matrix $\mu$: $1=1$, $-1=-1$.\\
Comparisons for matrix $\nu$: $-1\neq1$, $1\neq-1$.\\
The comparison $0$ and $0$ is not counted.

\begin{center}
\begin{tabular}{c|cccccccccc}
& & & & & & & & & & \\
\hline $S_1$ & 1 & 1 & -1 & 0 & 1 & -1 & -1 & -1 & -1 & 1 \\
$S_2$ & 0 & -1 & 1 & 1 & -1 & 1 & 1 & 1 & 1 & -1 \\
$S_3$ & 1 & -1 & -1 & 1 & -1 & -1 & 1 & -1 & 1 & 1 \\
$S_4$ & -1 & -1 & -1 & -1 & -1 & -1 & -1 & -1 & -1 & -1
\end{tabular}
\end{center}

\begin{center}
\begin{minipage}[b]{0.4\linewidth}
\begin{tabular}{c|cccc}
$\mu$ & $C_1$ & $C_2$ & $C_3$ & $C_4$ \\
\hline $C_1$ & \textcolor{red}{0.9} & 0 & 0.5 & 0.5 \\
$C_2$ & - & \textcolor{red}{0.9} & 0.5 & 0.3 \\
$C_3$ & - & - & 1 & 0.5 \\
$C_4$ & - & - & - & 1
\end{tabular}
\end{minipage}
\begin{minipage}[b]{0.4\linewidth}
\begin{tabular}{c|cccc}
$\nu$ & $C_1$ & $C_2$ & $C_3$ & $C_4$ \\
\hline $C_1$ & 0 & 0.8 & 0.4 & 0.4 \\
$C_2$ & - & 0 & 0.4 & 0.6 \\
$C_3$ & - & - & 0 & 0.5 \\
$C_4$ & - & - & - & 0
\end{tabular}
\end{minipage}
\end{center}
\end{frame}


\begin{frame}
Method \textbf{$\nu$-biased}.

\bigskip
Comparisons for matrix $\mu$: $1=1$, $-1=-1$.\\
Comparisons for matrix $\nu$: $0\neq0$, $-1\neq1$, $1\neq-1$.\\
The comparison $0$ and $0$ is counted for matrix $\nu$.

\begin{center}
\begin{tabular}{c|cccccccccc}
& & & & & & & & & & \\
\hline $S_1$ & 1 & 1 & -1 & 0 & 1 & -1 & -1 & -1 & -1 & 1 \\
$S_2$ & 0 & -1 & 1 & 1 & -1 & 1 & 1 & 1 & 1 & -1 \\
$S_3$ & 1 & -1 & -1 & 1 & -1 & -1 & 1 & -1 & 1 & 1 \\
$S_4$ & -1 & -1 & -1 & -1 & -1 & -1 & -1 & -1 & -1 & -1
\end{tabular}
\end{center}

\begin{center}
\begin{minipage}[b]{0.4\linewidth}
\begin{tabular}{c|cccc}
$\mu$ & $C_1$ & $C_2$ & $C_3$ & $C_4$ \\
\hline $C_1$ & \textcolor{red}{0.9} & 0 & 0.5 & 0.5 \\
$C_2$ & - & \textcolor{red}{0.9} & 0.5 & 0.3 \\
$C_3$ & - & - & 1 & 0.5 \\
$C_4$ & - & - & - & 1
\end{tabular}
\end{minipage}
\begin{minipage}[b]{0.4\linewidth}
\begin{tabular}{c|cccc}
$\nu$ & $C_1$ & $C_2$ & $C_3$ & $C_4$ \\
\hline $C_1$ & \textcolor{red}{0.1} & 0.8 & 0.4 & 0.4 \\
$C_2$ & - & \textcolor{red}{0.1} & 0.4 & 0.6 \\
$C_3$ & - & - & 0 & 0.5 \\
$C_4$ & - & - & - & 0
\end{tabular}
\end{minipage}
\end{center}
\end{frame}


\begin{frame}
Method \textbf{Balanced}.

\bigskip
We compute the methods $\mu$-biased and $\nu$-biased.\\
The elements for matrix $\mu$ are equal to: ($\mu_{\text{$\mu$-biased}}$ + $\mu_{\text{$\nu$-biased}}$)/2\\
The elements for matrix $\mu$ are equal to:($\nu_{\text{$\mu$-biased}}$ + $\nu_{\text{$\nu$-biased}}$)/2\\
The comparison $0$ and $0$ is distributed as half of it for matrix $\mu$\\and the other half for matrix $\nu$.

\begin{center}
\begin{tabular}{c|cccccccccc}
& & & & & & & & & & \\
\hline $S_1$ & 1 & 1 & -1 & 0 & 1 & -1 & -1 & -1 & -1 & 1 \\
$S_2$ & 0 & -1 & 1 & 1 & -1 & 1 & 1 & 1 & 1 & -1 \\
$S_3$ & 1 & -1 & -1 & 1 & -1 & -1 & 1 & -1 & 1 & 1 \\
$S_4$ & -1 & -1 & -1 & -1 & -1 & -1 & -1 & -1 & -1 & -1
\end{tabular}
\end{center}

\begin{center}
\begin{minipage}[b]{0.4\linewidth}
\begin{tabular}{c|cccc}
$\mu$ & $C_1$ & $C_2$ & $C_3$ & $C_4$ \\
\hline $C_1$ & \textcolor{red}{0.95} & 0 & 0.5 & 0.5 \\
$C_2$ & - & \textcolor{red}{0.95} & 0.5 & 0.3 \\
$C_3$ & - & - & 1 & 0.5 \\
$C_4$ & - & - & - & 1
\end{tabular}
\end{minipage}
\begin{minipage}[b]{0.4\linewidth}
\begin{tabular}{c|cccc}
$\nu$ & $C_1$ & $C_2$ & $C_3$ & $C_4$ \\
\hline $C_1$ & \textcolor{red}{0.05} & 0.8 & 0.4 & 0.4 \\
$C_2$ & - & \textcolor{red}{0.05} & 0.4 & 0.6 \\
$C_3$ & - & - & 0 & 0.5 \\
$C_4$ & - & - & - & 0
\end{tabular}
\end{minipage}
\end{center}
\end{frame}


\begin{frame}
Method \textbf{Weighted}.

\bigskip
We compute the method Unbiased.
Then we create new matrix $P$, which is the elementwise sum of the matrices $\mu$ and $\nu$ of the method Unbiased:
$P = \mu_{\text{unbiased}} + \nu_{\text{unbiased}}$.

\bigskip
$\mu_{\text{weighted}} := \mu_{\text{unbiased}} + \dfrac{\mu_{\text{unbiased}}}{P}(1 - P) = \dfrac{\mu_{\text{unbiased}}}{P}$
$\nu_{\text{weighted}} := \nu_{\text{unbiased}} + \dfrac{\nu_{\text{unbiased}}}{P}(1 - P) = \dfrac{\nu_{\text{unbiased}}}{P}$

\bigskip
Of course, for $P[i][j] \neq 0$. If $P[i][j] = 0$, then we assign $1/2$ for this element for both matrices.

\bigskip
We recall the matrices from method Unbiased:
\begin{center}
\begin{minipage}[b]{0.4\linewidth}
\begin{tabular}{c|cccc}
$\mu$ & $C_1$ & $C_2$ & $C_3$ & $C_4$ \\
\hline $C_1$ & 0.9 & 0 & 0.5 & 0.5 \\
$C_2$ & - & 0.9 & 0.5 & 0.3 \\
$C_3$ & - & - & 1 & 0.5 \\
$C_4$ & - & - & - & 1
\end{tabular}
\end{minipage}
\begin{minipage}[b]{0.4\linewidth}
\begin{tabular}{c|cccc}
$\nu$ & $C_1$ & $C_2$ & $C_3$ & $C_4$ \\
\hline $C_1$ & 0 & 0.8 & 0.4 & 0.4 \\
$C_2$ & - & 0 & 0.4 & 0.6 \\
$C_3$ & - & - & 0 & 0.5 \\
$C_4$ & - & - & - & 0
\end{tabular}
\end{minipage}
\end{center}
\end{frame}


\begin{frame}
Matrix $P$:
\begin{center}
\begin{tabular}{c|cccc}
$P$ & $C_1$ & $C_2$ & $C_3$ & $C_4$ \\
\hline $C_1$ & 0.9 & 0.8 & 0.9 & 0.9 \\
$C_2$ & - & 0.9 & 0.9 & 0.9 \\
$C_3$ & - & - & 1 & 1 \\
$C_4$ & - & - & - & 1
\end{tabular}
\end{center}

The new matrices for method Weighted:
\begin{center}
\begin{minipage}[b]{0.48\linewidth}
\begin{tabular}{c|cccc}
$\mu$ & $C_1$ & $C_2$ & $C_3$ & $C_4$ \\
\hline $C_1$ & 1 & 0 & 0.5556 & 0.5556 \\
$C_2$ & - & 1 & 0.5556 & 0.3333 \\
$C_3$ & - & - & 1 & 0.5 \\
$C_4$ & - & - & - & 1
\end{tabular}
\end{minipage}
\begin{minipage}[b]{0.48\linewidth}
\begin{tabular}{c|cccc}
$\nu$ & $C_1$ & $C_2$ & $C_3$ & $C_4$ \\
\hline $C_1$ & 0 & 1 & 0.4444 & 0.4444 \\
$C_2$ & - & 0 & 0.4444 & 0.6667 \\
$C_3$ & - & - & 0 & 0.5 \\
$C_4$ & - & - & - & 0
\end{tabular}
\end{minipage}
\end{center}
\end{frame}


\begin{frame}
\textit{Available types for ICrA calculations}

\bigskip
\textbf{Standard ICrA} -- apply ICrA to a single matrix or one set of matrices.\\
Ordered Pair -- apply ICrA to two matrices or two sets of matrices,\\
all three types allow ordered pair as input.

\bigskip
\textbf{Second Order ICrA} -- load at least 3 matrices of the same size, compute Standard ICrA for each of them,
take upper triangular matrix from $\mu$ and $\nu$ from each set, write as rows of new input matrices $\mu$ and $\nu$,
since we have two input matrices -- apply Standard ICrA for ordered pair.

\bigskip
\textbf{Aggregated ICrA} -- load at least 3 matrices of the same size, compute Standard ICrA for each of them,
then aggregate elementwise the resulting set of matrices $\mu$ and $\nu$: average, max-min, or min-max aggregation.
\end{frame}


\begin{frame}
\textit{Ordered pair $(\mu,\nu)$}

\bigskip
The ordered pair has four comparisons: greater than, less than, equal, incomparable.
Let two ordered pairs $(a_1,b_1)$ and $(a_2,b_2)$ be given. Then:
\begin{itemize}
\item $(a_1,b_1) > (a_2,b_2)$ when $a_1 \geq a_2$, $b_1 < b_2$ or $a_1 > a_2$, $b_1 \leq b_2$,
\item $(a_1,b_1) < (a_2,b_2)$ when $a_1 \leq a_2$, $b_1 > b_2$ or $a_1 < a_2$, $b_1 \geq b_2$,
\item $(a_1,b_1) = (a_2,b_2)$ when $a_1=a_2$, $b_1=b_2$,
\item incomparable: in the remaining cases, for example $a_1<a_2$, $b_1<b_2$.
\end{itemize}

\bigskip
Note that $a_1$, $b_1$, $a_2$, $b_2$ are integer or real values.\\
The standard method is a difference between each two values,\\
while this one is a comparison between each two pairs.
\end{frame}


\begin{frame}
\begin{thebibliography}{9}
\bibitem{amv2014} Atanassov K., D. Mavrov, V. Atanassova, InterCriteria Decision Making: A New Approach for Multicriteria Decision Making, Based on Index Matrices and Intuitionistic Fuzzy Sets, \textit{Issues in Intuitionistic Fuzzy Sets and Generalized Nets}, 11, 1-8, 2014.
\bibitem{atanassov2014} Atanassov K., \textit{Index Matrices: Towards an Augmented Matrix Calculus}, Studies in Computational Intelligence, 573, 2014.
\bibitem{atanassov2012} Atanassov K., \textit{On Intuitionistic Fuzzy Sets Theory}, Springer, Berlin, 2012.
\bibitem{icrasiteofficial} \url{http://intercriteria.net/software/}
\bibitem{icrasite} \url{http://justmathbg.info/icradata.html}
\end{thebibliography}
\end{frame}


\begin{frame}
\begin{center}
Thank you for the attention!
\end{center}

\vfill
This application has been developed with the partial financial support of:
Project No. DFNI-I-02-5 ``InterCriteria Analysis: A New Approach to Decision Making'',
funded by the National Science Fund of Bulgaria.
\end{frame}


\end{document}

