%%%
%%% Second order InterCriteria Analysis
%%%

\documentclass[12pt, a4paper]{article}


\begin{document}

\title{\textbf{\Large An idea for second order InterCriteria Analysis}}
\author{Explanations by Nikolay Ikonomov}
\date{\today}
\maketitle


\section{An idea for the method by Olympia Roeva}

This section describes some ideas and concepts about second order InterCriteria Analysis (ICrA),
where the third dimension is viewed as a parameter which changes its value.
Real world example (data based on genetic algorithms) provided by Olympia Roeva.

Suppose we have a matrix with 5 criteria and 30 objects.
\[ \begin{array}{c|cccc}
& O_1 & O_2 & \ldots & O_{30} \\
\hline
C_1 & C_1(O_1) & C_1(O_2) & \ldots & C_1(O_{30}) \\
C_2 & C_2(O_1) & C_2(O_2) & \ldots & C_2(O_{30}) \\
C_3 & C_3(O_1) & C_3(O_2) & \ldots & C_3(O_{30}) \\
C_4 & C_4(O_1) & C_4(O_2) & \ldots & C_4(O_{30}) \\
C_5 & C_5(O_1) & C_5(O_2) & \ldots & C_5(O_{30}) \\
\end{array} \]
Applying ICrA gives two matrices with size 5x5, one for $\mu$ and another one for $\nu$.
Since the matrices are symmetric, we need only the elements from the lower triangular matrix
(or upper, they are the same elements).

Take these values as vectors, result is two vectors with size 10.
\[ \begin{array}{c|ccccccc|cccc}
V_\mu & & & & & & & V_\nu & & & & \\
\cline{1-5} \cline{8-12}
& V_{\mu,1} & V_{\mu,2} & \ldots & V_{\mu,10} & \mbox{ } & \mbox{ } & & V_{\nu,1} & V_{\nu,2} & \ldots & V_{\nu,10} \\
\end{array} \]
Suppose we have a parameter, which changes from 0.01 to 0.1 with step 0.01, the parameter has 100 different values.
Apply the above for each value, result is 100 different matrices $\mu$ and $\nu$.

Take only the values from the lower triangular matrix from each two matrices.
\[ \begin{array}{c|ccccccc|cccc}
V_\mu & o_1 & o_2 & \ldots & o_{10} & & & V_\nu & o_1 & o_2 & \ldots & o_{10} \\
\cline{1-5} \cline{8-12}
c_1 & V_{\mu,1,1} & V_{\mu,1,2} & \ldots & V_{\mu,1,10} & \mbox{ } & \mbox{ } & c_1 & V_{\nu,1,1} & V_{\nu,1,2} & \ldots & V_{\nu,1,10} \\
c_2 & V_{\mu,2,1} & V_{\mu,2,2} & \ldots & V_{\mu,2,10} & \mbox{ } & \mbox{ } & c_2 & V_{\nu,2,1} & V_{\nu,2,2} & \ldots & V_{\nu,2,10} \\
\vdots & \vdots & \vdots & \ddots & \vdots &   &   & \vdots & \vdots & \vdots & \ddots & \vdots \\
c_{50} & V_{\mu,50,1} & V_{\mu,50,2} & \ldots & V_{\mu,50,10} & \mbox{ } & \mbox{ } & c_{50} & V_{\nu,50,1} & V_{\nu,50,2} & \ldots & V_{\nu,50,10} \\
\vdots & \vdots & \vdots & \ddots & \vdots &   &   & \vdots & \vdots & \vdots & \ddots & \vdots \\
c_{100} & V_{\mu,100,1} & V_{\mu,100,2} & \ldots & V_{\mu,100,10} & \mbox{ } & \mbox{ } & c_{100} & V_{\nu,100,1} & V_{\nu,100,2} & \ldots & V_{\nu,100,10}
\end{array} \]
We can view this data as 100 criteria and 10 objects.

Now apply ICrA to these two matrices, loaded as ordered pair, and receive two matrices $\mu$ and $\nu$.
Again, take only the values from the lower triangluar matrix, result is two vectors of size 100.
The result shows the relation between the 100 different values of the parameter.

First, we had objects and criteria, after ICrA, the result is viewed as new objects,
each change of the parameter is viewed as a criteria, so that second ICrA can be applied.

\bigskip
What this algorithm does is the following:
\begin{enumerate}
\item load matrix of size 5x30 (5 criteria and 30 objects)
\item apply ICrA and receive two matrices of size 5x5
\item take the unique values as two vectors of size 10
\item change a parameter 100 times and apply points 1-3
\item write data as two matrices of size 100x10 (100 criteria and 10 objects)
\item apply ICrA and receive two matrices of size 100x100
\item take the unique values as two vectors of size 100
\end{enumerate}

ICrA was applied two times, therefore second order ICrA.
The change of the parameter gives us 100 different states,
this might be viewed as time, let's call it depth.
The 30 objects can be viewed as width, and the 5 criteria as height.
Therefore we have 30x5x100 (WxHxD), three dimensional ICrA.



\section{An idea for the method by Velichka Traneva}

This section describes some ideas and concepts about three dimensional InterCriteria Analysis, where $T$ is the third fixed scale and $t_g$ is its element. For example, index set $T$ can be interpreted as a time-scale and its elements $t_g$ -- as time-moments.

Suppose we have a matrix with 5 criteria and 30 objects.
\[ \begin{array}{c|cccc}
& O_1 & O_2 & \ldots & O_{30} \\
\hline
C_1 & C_1(O_1) & C_1(O_2) & \ldots & C_1(O_{30}) \\
C_2 & C_2(O_1) & C_2(O_2) & \ldots & C_2(O_{30}) \\
C_3 & C_3(O_1) & C_3(O_2) & \ldots & C_3(O_{30}) \\
C_4 & C_4(O_1) & C_4(O_2) & \ldots & C_4(O_{30}) \\
C_5 & C_5(O_1) & C_5(O_2) & \ldots & C_5(O_{30}) \\
\end{array} \]
Applying ICrA gives two matrices with size 5x5, one for $\mu$ and another one for $\nu$.
Let $t_k$ change 7 times, $t_1, t_2, \ldots, t_7$. Apply ICrA for each value.
Result is 7 matrices for $\mu$ and $\nu$.
\[ \begin{array}{c|cccc cc c|cccc}
\mu(t_1) & & & & & & & \nu(t_1) & & & & \\
\cline{1-5} \cline{8-12}
& \mu_{1,1,t_1} & \mu_{1,2,t_1} & \ldots & \mu_{1,5,t_1} & \mbox{ } & \mbox{ } & & \nu_{1,1,t_1} & \nu_{1,2,t_1} & \ldots & \nu_{1,5,t_1} \\
& \mu_{2,1,t_1} & \mu_{2,2,t_1} & \ldots & \mu_{2,5,t_1} & \mbox{ } & \mbox{ } & & \nu_{2,1,t_1} & \nu_{2,2,t_1} & \ldots & \nu_{2,5,t_1} \\
& \vdots & \vdots & \ddots & \vdots & \mbox{ } & \mbox{ } & & \vdots & \vdots & \ddots & \vdots \\
& \mu_{5,1,t_1} & \mu_{5,2,t_1} & \ldots & \mu_{5,5,t_1} & \mbox{ } & \mbox{ } & & \nu_{5,1,t_1} & \nu_{5,2,t_1} & \ldots & \nu_{5,5,t_1}
\end{array} \]
\[ \begin{array}{c|cccc cc c|cccc}
\mu(t_2) & & & & & & & \nu(t_2) & & & & \\
\cline{1-5} \cline{8-12}
& \mu_{1,1,t_2} & \mu_{1,2,t_2} & \ldots & \mu_{1,5,t_2} & \mbox{ } & \mbox{ } & & \nu_{1,1,t_2} & \nu_{1,2,t_2} & \ldots & \nu_{1,5,t_2} \\
& \mu_{2,1,t_2} & \mu_{2,2,t_2} & \ldots & \mu_{2,5,t_2} & \mbox{ } & \mbox{ } & & \nu_{2,1,t_2} & \nu_{2,2,t_2} & \ldots & \nu_{2,5,t_2} \\
& \vdots & \vdots & \ddots & \vdots & \mbox{ } & \mbox{ } & & \vdots & \vdots & \ddots & \vdots \\
& \mu_{5,1,t_2} & \mu_{5,2,t_2} & \ldots & \mu_{5,5,t_2} & \mbox{ } & \mbox{ } & & \nu_{5,1,t_2} & \nu_{5,2,t_2} & \ldots & \nu_{5,5,t_2}
\end{array} \]
\[ \begin{array}{ccc}
\vdots & \hspace{7cm} & \vdots \\
\end{array} \]
\[ \begin{array}{c|cccc cc c|cccc}
\mu(t_7) & & & & & & & \nu(t_7) & & & & \\
\cline{1-5} \cline{8-12}
& \mu_{1,1,t_7} & \mu_{1,2,t_7} & \ldots & \mu_{1,5,t_7} & \mbox{ } & \mbox{ } & & \nu_{1,1,t_7} & \nu_{1,2,t_7} & \ldots & \nu_{1,5,t_7} \\
& \mu_{2,1,t_7} & \mu_{2,2,t_7} & \ldots & \mu_{2,5,t_7} & \mbox{ } & \mbox{ } & & \nu_{2,1,t_7} & \nu_{2,2,t_7} & \ldots & \nu_{2,5,t_7} \\
& \vdots & \vdots & \ddots & \vdots & \mbox{ } & \mbox{ } & & \vdots & \vdots & \ddots & \vdots \\
& \mu_{5,1,t_7} & \mu_{5,2,t_7} & \ldots & \mu_{5,5,t_7} & \mbox{ } & \mbox{ } & & \nu_{5,1,t_7} & \nu_{5,2,t_7} & \ldots & \nu_{5,5,t_7}
\end{array} \]
We apply an aggregation at each value $(i,j)$, thus creating an aggregated matrix for $\mu$ and aggregated matrix for $\nu$.

\begin{itemize}
\item MaxMin aggregation: that is, create new matrix of size 5x5,
compute each value by taking all 7 values (at the respective indices) from the matrices $\mu(t_1),\ldots,\mu(t_7)$:
\[ \mu_{i,j,R} = \max(\mu_{i,j,t_1},\mu_{i,j,t_2},\mu_{i,j,t_3},\mu_{i,j,t_4},\mu_{i,j,t_5},\mu_{i,j,t_6},\mu_{i,j,t_7}) \]
This is the result matrix for $\mu$. Now create one more matrix of size 5x5, take values from $\nu(t_1),\ldots,\nu(t_7)$:
\[ \nu_{i,j,R} = \min(\nu_{i,j,t_1},\nu_{i,j,t_2},\nu_{i,j,t_3},\nu_{i,j,t_4},\nu_{i,j,t_5},\nu_{i,j,t_6},\nu_{i,j,t_7}) \]
And this is the reult matrix for $\nu$.

\item MinMax aggregation: find the minimum values for $\mu$ and the maximum values for $\nu$:
\[ \mu_{i,j,R} = \min(\mu_{i,j,t_1},\mu_{i,j,t_2},\mu_{i,j,t_3},\mu_{i,j,t_4},\mu_{i,j,t_5},\mu_{i,j,t_6},\mu_{i,j,t_7}) \]
\[ \nu_{i,j,R} = \max(\nu_{i,j,t_1},\nu_{i,j,t_2},\nu_{i,j,t_3},\nu_{i,j,t_4},\nu_{i,j,t_5},\nu_{i,j,t_6},\nu_{i,j,t_7}) \]

\item Average aggregation: find the arithmetic mean values for $\mu$ and for $\nu$:
\[ \mu_{i,j,R} = (\mu_{i,j,t_1} + \mu_{i,j,t_2} + \mu_{i,j,t_3} + \mu_{i,j,t_4} + \mu_{i,j,t_5} + \mu_{i,j,t_6} + \mu_{i,j,t_7})/7 \]
\[ \nu_{i,j,R} = (\nu_{i,j,t_1} + \nu_{i,j,t_2} + \nu_{i,j,t_3} + \nu_{i,j,t_4} + \nu_{i,j,t_5} + \nu_{i,j,t_6},\nu_{i,j,t_7})/7 \]
\end{itemize}

After we apply an aggregation, we receive two matrices of size 5x5, one for $\mu$ and another for $\nu$.
\[ \begin{array}{c|cccc cc c|cccc}
\mu(R) & & & & & & & \nu(R) & & & & \\
\cline{1-5} \cline{8-12}
& \mu_{1,1,R} & \mu_{1,2,R} & \ldots & \mu_{1,5,R} & \mbox{ } & \mbox{ } & & \nu_{1,1,R} & \nu_{1,2,R} & \ldots & \nu_{1,5,R} \\
& \mu_{2,1,R} & \mu_{2,2,R} & \ldots & \mu_{2,5,R} & \mbox{ } & \mbox{ } & & \nu_{2,1,R} & \nu_{2,2,R} & \ldots & \nu_{2,5,R} \\
& \vdots & \vdots & \ddots & \vdots & \mbox{ } & \mbox{ } & & \vdots & \vdots & \ddots & \vdots \\
& \mu_{5,1,R} & \mu_{5,2,R} & \ldots & \mu_{5,5,R} & \mbox{ } & \mbox{ } & & \nu_{5,1,R} & \nu_{5,2,R} & \ldots & \nu_{5,5,R}
\end{array} \]
These matrices $\mu(R)$ and $\nu(R)$ can be considered as the final result.


\end{document}

