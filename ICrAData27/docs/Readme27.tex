%%%
%%% Readme for ICrAData v2.7
%%%
%%% Nikolay Ikonomov
%%%
%%% 22 March 2025
%%%

\documentclass[11pt, a4paper]{article}
\usepackage[a4paper, margin=3cm]{geometry}
\usepackage{url}

\begin{document}

\title{\textbf{ICrAData v2.7}}
\author{\textit{InterCriteria Analysis algorithms:}\\
Krassimir Atanassov, Olympia Roeva, Velichka Traneva, Peter Vassilev\\
\textit{Software programming:}\\
Nikolay Ikonomov}
\date{22 March 2025}
\maketitle

% \tableofcontents

\newpage
The name of the application ICrAData stands for InterCriteria Analysis Data.
For basic information about the algorithms see the presentation in \verb|docs| directory,
for details see the articles in \verb|external| directory.


\section{Application}

The application is written in \verb|C/FLTK|.
Start the application from \verb|ICrAData.sh| on Linux, \verb|ICrAData.exe| on Windows,
alternatively try \verb|ICrADataMSVC.exe| on Windows.

The basic example is already loaded:
\begin{verbatim}
x;E;F;G;H;I
A;6;5;3;7;6
B;7;7;8;1;3
C;4;3;5;9;1
D;4;5;6;7;8
\end{verbatim}
For the example to process, select Semicolon separator and mark the Headers.
Then click the button Analysis to make the calculations and view the plot.

If headers exist, note that the first element on the first row must be present, even though it does not affect the data.
Otherwise the application reports error ``All column sizes per row must match (including the header)''.


\section{Algorithms}

There are Variants and Methods. Variants are the algorithms by which the matrix is processed.
Usually that is a single matrix, also known as Standard ICrA Method.

The other methods must be applied to at least three matrices. There is matrix count MatCnt option for that.

\medskip
\textit{Standard ICrA Method:} applies the base algorithm (ICrA Variant) over a single matrix, and displays the result.

\medskip
\textit{Aggregated ICrA Method:} requires at least three matrices, the input matrix is split by the MatCnt option.
The base algorithm is applied over each matrix.
Then is applied an aggregation over the matrix count: average of all elements at each $(i,j)$ index,
maximum/minimum of all elements at each $(i,j)$ index.

\medskip
\textit{Criteria Pair ICrA Method:} requires at least three matrices, the input matrix is split by the MatCnt option.
The base algorithm is applied over each matrix. This result is written as rows of two new matrices, one for $\mu$ and one for $\nu$.
The intermediary step is two matrices with number of rows equal the matrix count,
number of columns equal the number of elements of upper triangular matrix from the base algorithm.

A criteria pair (special method) is applied to these two matrices, which results in a new ICrA matrix, that is displayed.
This special method applies the base algorithm over an Ordered Pair (this functionality is used only here).

\medskip
ICrAData v1.8 has the capability to load data files as Ordered Pair, which is not yet implemented in the 2.x branch.


\section{Result}

Table view can be changed by the drop down menus.
The values for Alpha and Beta refresh the tables and the plot.
Table and plot colors:
\begin{itemize}
\item $\mu > \alpha$ and $\nu < \beta$ -- Positive Consonance -- green,
\item $\mu < \beta$ and $\nu > \alpha$ -- Negative Consonance -- red,
\item all other cases -- Dissonance -- magenta.
\end{itemize}
The Export button allows saving the input or result matrix in several different ways:
\begin{itemize}
\item input -- save the input matrix with these parameters;
\item $\mu$/$\nu$ output -- $\mu$ data is saved in the upper triangular part of the result matrix,
$\nu$ data is saved in the lower triangular part of the result matrix,
this option saves the result matrix the way it is stored in memory;
\item $(\mu;\nu)$ table -- save the result matrix as a full mirror matrix -- values for $(\mu;\nu)$
are repeated in the upper/lower part of the saved matrix file;
\item $\mu$ table -- mirror matrix for $\mu$ values;
\item $\nu$ table -- mirror matrix for $\nu$ values;
\item vector upper -- save the result matrix as a vector -- headers, values, indexes -- per each cell per row,
iterate over the upper triangular part of the result matrix;
\item vector lower -- save the result matrix as a vector -- headers, values, indexes -- per each cell per row,
iterate over the lower triangular part of the result matrix, different ordering of the elements compared to vector upper.
\end{itemize}


\section{Plot}

Top of the right panel:
\begin{itemize}
\item Circle size -- size of the plot points.
\item Color/Marks/Grid/Text are displayed in the right panel.
\item Button TeX saves the plot in TeX format with 9 parameters:
Alpha, Beta, Points, Matrix break, Jump after matrix break, Color, Marks, Grid, Text.
If there are too many points on the plot, the TeX file might be so large, that it does not compile.
Then increase the Jump to 50 so that TeX makes the PDF file.
\item Button PNG saves the plot in PNG format as displayed in the right panel.
\end{itemize}
Bottom of the right panel:
\begin{itemize}
\item Clock/Matrix -- when matrix rows or columns are greater than this value, show a clock during calculations and apply Jump for the plot.
\item Jump/Matrix -- skip this many elements when drawing the plot. Very useful for big data, change it to redraw the plot.
\item Threads -- maximum count of CPU threads to use for the calculations. Affects the speed of the calculations, this option is always enabled.
\item Light/Dark -- default is dark mode, light mode is useful for PNG image of the plot, since the background is white.
\end{itemize}

A comment is required here -- the multi-threaded code is always enabled, for all calculations, regardless of the value of Clock/Matrix --
this option is only for showing the clock, otherwise the user interface is blocked until the calculations complete.


\section{High speed calculations}

The high performance code uses a very famous library -- GNU OpenMP. This library utilizes the CPU fully,
but if you start an application, CPU usage will be lowered, and your computer will be usable normally.
This way calculations will take longer to complete.

There are two possible problems:

1) If the input matrix is too big, and calculations complete, then displaying the plot and the tables might take too long -- hours.
Therefore, by default, the tables are hidden for value greater than Clock/Matrix, and only the plot is displayed.
If the plot takes more than one minute to draw, it is very likely that the tables will take forever to draw.
In this case, use the Export button to save the tables to hard drive, and then open them with MS Excel/LibreOffice Calc.

2) There is a technology known as pagefile on Windows and swapfile on Linux.
This technology moves stuff from RAM to HDD if the applications require more than your physical memory.
As expected, this locks the HDD fully, and your computer hangs.
There is no workaround, except deactivating this technology, or bying more RAM.

Windows -- Search -- Advanced System Settings -- Performance -- Settings -- Advanced -- Virtual Memory --
Change -- No paging file -- Set for all HDDs -- reboot the computer

Linux -- swapoff -a -- vim /etc/fstab -- comment \# all lines containing swap -- save file -- rm /swapfile to clean up

Also, you can reduce the size of the input matrix :) hint, hint :)


\section{Acknowledgements}

This software application has been developed with the partial financial support of:

Changes in versions from \textbf{1.3} to \textbf{2.6} have been implemented for
project \textbf{DN 17/06 ``A New Approach, Based on an Intercriteria Data Analysis,
to Support Decision Making in `in silico' Studies of Complex Biomolecular Systems''},
funded by the National Science Fund of Bulgaria.

Changes in versions from \textbf{0.9.6} to \textbf{1.2} have been implemented for
project \textbf{DFNI-I-02-5 ``InterCriteria Analysis: A New Approach to Decision Making''},
funded by the National Science Fund of Bulgaria.


\section{Credits}

This software application contains statically compiled code from
FLTK 1.4.2 / 23-Feb-2025 \url{https://www.fltk.org/}.


% \newpage
\section{Changelog}

\begin{itemize}
\item Version 2.7 (22 March 2025) [C/FLTK]
\begin{itemize}
\item Synchronized open/save file parameters with ICrAData v1.9.1
\item Application saves text files by default
\item Plot image circles are now always drawn properly [from 2.6]
\end{itemize}

\item {\it Version 1.9.1 (November 16, 2024) [Java]}
\begin{itemize}\it
\item Improved open/save file parameters detection
\item Enabled HiDPI when saving plot graphic and screenshot
\end{itemize}

\item {\it Version 1.9 (October 5, 2024) [Java]}
\begin{itemize}\it
\item Reworked export options for tables
\item Fixed division by zero in Weighted
\item Source code refactored
\end{itemize}

\item Version 2.6 (March 19, 2023) [C/FLTK]
\begin{itemize}
\item Light theme by default
\item PNG plot image is now saved properly [from 2.2]
\item Plot image circles are now filled
\end{itemize}

\item Version 2.5 (May 9, 2021) [C/FLTK]
\begin{itemize}
\item Re-enabled compilation by MSVC [removed in 2.3]
\item Table export is now working properly [from 2.1]
\end{itemize}

\item Version 2.4 (January 10, 2021) [C/FLTK]
\begin{itemize}
\item Clarified display/export tables
\item Removed config.ini
\end{itemize}

\item {\it Version 1.8 (January 9, 2021) [Java]}
\begin{itemize}\it
\item Interface optimized to reflect version 2.3
\end{itemize}

\item Version 2.3 (August 16, 2020) [C/FLTK]
\begin{itemize}
\item Added proper extension when saving files
\item Added file filters to the file chooser
\item Scale and colors can be changed from config.ini
\end{itemize}

\item Version 2.2 (June 24, 2020) [C/FLTK]
\begin{itemize}
\item Enabled native file chooser for file operations
\item PNG plot image is now saved in square size
\item Scale and colors can be changed from the application
\end{itemize}

\newpage
\item Version 2.1 (June 21, 2020) [C/FLTK]
\begin{itemize}
\item Added multi-threaded code for much faster calculations
\item Interface optimized to reflect version 1.7
\end{itemize}

\item {\it Version 1.7 (June 10, 2020) [Java]}
\begin{itemize}\it
\item Algorithms optimized to use less memory
\item Interface optimized to reflect version 2.0
\end{itemize}

\item Version 2.0 (January 2, 2020) [C/FLTK]
\begin{itemize}
\item Algorithms optimized to use less memory
\item Application rewrite from the codebase of ICrAData v1.6
\end{itemize}
\end{itemize}


\section{Download}

Download the application from these links:

\url{https://intercriteria.net/software/}

\url{https://justmathbg.info/icradata2.html}


\end{document}

